\documentclass[a4paper,10pt]{extarticle}
\usepackage[T1]{fontenc}
\usepackage[utf8]{inputenc}
\usepackage{lmodern}
\usepackage{ngerman}
\usepackage[fleqn]{amsmath}
\usepackage{amssymb}
\usepackage{amsmath}
\usepackage{titlesec}
\usepackage{siunitx}
\usepackage{physics}


\title{Übungsblatt 6}
\author{Lennard Behrens (3200335), Gabriel Kraus (3208414), Ole Schmidt(3206580)}

\begin{document}
\maketitle

\section*{Aufgabe 3}
  \subsection*{Aufgbe 3a}
  Der Koordinatenursprund liegt im Punkt $A$, mit der positiven $x-$Achse nach unten zeigend. \\ 
  Die Höhe des Balls läßt sich folgendermaßen beschreiben:
  \begin{align*}
  h_B &= h_0 + v_0 t + \frac{1}{2}gt^2 \\
  h_0 &= 10 \, \mbox{m} \\ 
  v_0 &= 0 \, \frac{\mbox{m}}{\mbox{s}} \\ \\
  h_A &= h'_0 + v_A t \\
  h'_0 &= -2 \frac{\mbox{m}}{\mbox{s}}
  \end{align*}
  Durch gleichsetzen der beiden Gleichungen kann die Zeit des Zusammentreffens ermittelt werden.
  \begin{align*}
  h_0 + \frac{1}{2}gt^2 &= h'_0 + v_A t \\
  0 &= t^2 - \frac{2 v_A}{g} t - \frac{2(h_0 - h'_0)}{g} \\
  t_1 &= 1.826 \, \mbox{s} \\
  t_2 &= -2.233 \, \mbox{s} \quad \rightarrow \mbox{physikalisch nicht sinnvoll}\\
  h_{(1.826)} &= 26.35\, \mbox{m}
  \end{align*}

  \subsection*{Aufgabe 3b}
  Formel aus Aufgabe 1, mit der Annahme, dass der Aufzug deutlich schwerer als der Ball ist.
  \begin{align*}
  v_{B(t_1)} &= v_0 + gt = 17.9 \, \frac{\mbox{m}}{\mbox{s}} \\
  v'_2 &= \frac{(m_2 - m_1)v_2 + 2 m_1 v_1}{m_1 + m_2} \\
  m_1 &\gg m_2 \\
  v'_2 &= 2v_1 - v_2 = 2(-2) - 17.9 \, \frac{\mbox{m}}{\mbox{s}} \\
  v'_2 &= -21.9 \frac{\mbox{m}}{\mbox{s}} \\
  \end{align*}
  Die maximale Höhe ist erreicht, wenn die gesamte kinetische Energie des Balls in potentielle Energie umgewandelt wurde.
  \begin{align*}
  \frac{1}{2} m {v'_2}^2 &= m g h'_{max} \\
  h'_{max} &= \frac{{v'_2}^2}{2g} = -24.45 \, \mbox{m} \\
  h_{max} &= h_0 + h'_{max} = 26.35 \, \mbox{m} - 24.45 \, \mbox{m} \\
  h_{max} &= 1.897 \, \mbox{m}
  \end{align*}

  \subsection*{Aufgabe 3c}
  \begin{align*}
  t_1 &= 1.825 \, \mbox{s} \\ \\
  t_2 &= \mbox{bis der Ball $h_{max}$ erreicht} \\
  h_{(t)} &= h_0 + v_0t - \frac{1}{2}gt^2 \\
  \dot{h_{(t)}} &= v_{(t)} = v_0 - gt \quad v_0 = 0 \mbox{ bei } h_{max} \\
  t &= \frac{v_0}{g} = 2.232 \, \mbox{s} \\ \\
  h_B{(t)} \mbox{ ab } t_2 &= h_{max} + \frac{1}{2}gt^2 \\
  h_A{(t)} \mbox{ ab } t_2 &= h'_0 = v_A(t + t_1 + t_2) \\
  \end{align*}
  Durch gleichsetzen der beiden Gleichungen kommt man auf den Zeitpunkt (analog zu Teil a der Aufgabe)
  \begin{align*}
  t_3 &= 1.825 \, \mbox{s}
  h{(t_1 + t_2 + t_3)} &= 30 \, \mbox{m} - 2 \, \frac{\mbox{m}}{\mbox{s}} (1.825 + 2.232 + 1.825)\, \mbox{s} \\
  h{(t_1 + t_2 + t_3)} &= 18.236 \, \mbox{m}
  \end{align*}
\section*{Aufgabe 4}
  \subsection*{Aufgabe 4a}
  Da der Gesamtimpuls vor der Explosion $0$ war, gilt:
  \begin{align*}
  m_1 v_1 + m_2 v_2 &= - m_3 v_3 \\
  v_3 &= - \frac{m_1}{m_3} v_1 - \frac{m_2}{m_3} v_2 = -3 \begin{pmatrix} -2 \\ 0 \\ 2 \end{pmatrix} -2 \begin{pmatrix} 0 \\ 3 \\ -3 \end{pmatrix} = \begin{pmatrix} x \\ y \\ z \end{pmatrix} \\
  &\rightarrow v_3 = \begin{pmatrix} 6 \\ -6 \\ 0 \end{pmatrix}
  \end{align*}

  \subsection*{Aufgabe 4b}
  Die Gesamtenergie, die frei wird, ist die Summe, der kinetischen Energien, der 3 Teile.
  \begin{align*}
  m_1 &= \frac{3}{6}m, \quad
  m_2 = \frac{2}{6}m, \quad
  m_3 = \frac{1}{6}m \\
  E_{ges} &= \frac{1}{2}(m_1 v_1^2 + m_2 v_2^2 + m_3 v_3^2) \\
  E_{ges} &= 3.3 \, \mbox{J} 
  \end{align*}

\end{document}