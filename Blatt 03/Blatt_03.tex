\documentclass[a4paper,10pt]{extarticle}
\usepackage[T1]{fontenc}
\usepackage[utf8]{inputenc}
\usepackage{lmodern}
\usepackage{ngerman}
\usepackage[fleqn]{amsmath}
\usepackage{amssymb}
\usepackage{amsmath}
\usepackage{titlesec} 

\title{Übungsblatt 3}
\author{Lennard Behrens (3200335), Gabriel Kraus (3208414)}

\begin{document}
\maketitle

\section*{Aufgabe 5}
\begin{align*}
  \vec{a} \times (\vec{b} \times \vec{c}) &=
  \vec{a} \times 
  \left(\begin{array}{c}
    b_2 c_3-b_3 c_2\\
    b_3 c_1-b_1 c_3\\
    b_1 c_2-b_2 c_1\\
  \end{array}\right)
  \\&= 
  \left(\begin{array}{c}
    a_2 b_1 c_2+a_3 b_1 c_3-a_2 b_2 c_1-a_3 b_3 c_1\\
    a_1 b_2 c_1+a_3 b_2 c_3-a_1 b_1 c_2-a_3 b_3 c_2\\
    a_1 b_3 c_1+a_2 b_3 c_2-a_1 b_1 c_3-a_2 b_2 c_3
  \end{array}\right)\\
  (\vec{a}\cdot\vec{c})\vec{b}-(\vec{a}\cdot\vec{b})\vec{c}
  &= -(a_1 b_1 + a_2 b_2 + a_3 b_3)\{c_1, c_2, c_3\} + (a_1 c_1 + a_2 c_2 + a_3 c_3)\{b_1, 
  b_2, b_3\}\\
  &=\left(\begin{array}{c}
    a_1 c_1 b_1 + a_2 c_2 b_1 + a_3 c_3 b_1 - a_1 b_1 c_1 - a_2 b_2 c_1 - a_3 b_3 c_1\\
    a_1 c_1 b_2 + a_2 c_2 b_2 + a_3 c_3 b_2 - a_1 b_1 c_2 - a_2 b_2 c_2 - a_3 b_3 c_2\\
    a_1 c_1 b_3 + a_2 c_2 b_3 + a_3 c_3 b_3 - a_1 b_1 c_3 - a_2 b_2 c_3 - a_3 b_3 c_3\\
  \end{array}\right)
  \\&=\left(\begin{array}{c}
    a_2 c_2 b_1 + a_3 c_3 b_1 - a_2 b_2 c_1 - a_3 b_3 c_1\\
    a_1 c_1 b_2 + a_3 c_3 b_2 - a_1 b_1 c_2 - a_3 b_3 c_2\\
    a_1 c_1 b_3 + a_2 c_2 b_3 - a_1 b_1 c_3 - a_2 b_2 c_3\\
  \end{array}\right)
\end{align*}
Die Ausdrücke sind identisch.
\begin{align*}
  (\vec{a}\times\vec{b})\times\vec{c} &=
  \left(\begin{array}{c}
    a_2 b_1 c_2 + a_3 b_1 c_3 - a_1 b_2 c_2 - a_1 b_3 c_3\\
    a_1 b_2 c_1 + a_3 b_2 c_3 - a_2 b_3 c_3 - a_2 b_1 c_1\\
    a_1 b_3 c_1 + a_2 b_3 c_2 - a_3 b_2 c_2 - a_3 b_1 c_1 
  \end{array}\right) \\
  &= -\vec{c}\times(\vec{a}\times\vec{b})\\
  &= -((\vec{c}\cdot\vec{b})\vec{a} - (\vec{c}\cdot\vec{a})\vec{b})\\
  &= (\vec{c}\cdot\vec{a})\vec{b} - (\vec{c}\cdot\vec{b})\vec{a}
\end{align*}
Man erkennt, dass $\vec{a} \times (\vec{b} \times \vec{c})$
in der Ebene liegt, die von $\vec{b}$ und $\vec{c}$ aufgespannt wird.
Analog dazu liegt $(\vec{a}\times\vec{b})\times\vec{c}$ in der Ebene, die von 
$\vec{a}$ und $\vec{b}$ aufgespannt wird.
\end{document}
