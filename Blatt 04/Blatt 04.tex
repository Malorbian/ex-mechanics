\documentclass[a4paper,10pt]{extarticle}
\usepackage[T1]{fontenc}
\usepackage[utf8]{inputenc}
\usepackage{lmodern}
\usepackage{ngerman}
\usepackage[fleqn]{amsmath}
\usepackage{amssymb}
\usepackage{amsmath}
\usepackage{titlesec}
\usepackage{siunitx}
\usepackage{physics}


\title{Übungsblatt 4}
\author{Lennard Behrens (3200335), Gabriel Kraus (3208414)}

\begin{document}
\maketitle

\section*{Aufgabe 2}
\subsection*{Aufgabe 2a}
Ermittlung der Einheitsvektoren
\begin{align*}
x &= \cos(\varphi), \qquad y = \sin(\varphi), \qquad z = z \\
\varphi &= \omega t \\ \\
\pdv{\vec{r}}{\varrho} &= \cos(\omega t ) \hat{x} + \sin(\omega t ) \hat{y} \\ 
| \pdv{\vec{r}}{\varrho} | &= \sqrt{\cos(\omega t ) \hat{x} + \sin(\omega t ) \hat{y}} = 1 \\ 
\hat{\varrho}_{(t)} &= \frac{\pdv{\vec{r}}{\varrho}}{ | \pdv{\vec{r}}{\varrho} | } = \cos(\omega t ) \hat{x} + \sin(\omega t ) \hat{y} \\ \\
\pdv{\vec{r}}{\varphi} &= \varrho (-\sin(\omega t ) \hat{x} + \cos(\omega t ) \hat{y}) \\ 
| \pdv{\vec{r}}{\varphi} | &= \sqrt{\varrho (-\sin(\omega t ) \hat{x} + \cos(\omega t ) \hat{y})} = \varrho \\ 
\hat{\varphi}_{(t)} &= \frac{\pdv{\vec{r}}{\varphi}}{ | \pdv{\vec{r}}{\varphi} | } = -\sin(\omega t ) \hat{x} + \cos(\omega t ) \hat{y} \\ \\
\hat{z} &= \hat{z}
\end{align*}
Daraus ergeben sich die Zylinderkoordinaten: 
\begin{align*}
\vec{r_{(t)}} = \varrho_0 \hat{\varrho} + v_z t \hat{z}
\end{align*}

\subsection*{Aufgabe 2b}
$\vec{r}_{(t)}$ nach der Zeit ableiten.
\begin{align*}
\dot{\vec{r}}_{(t)} &= \varrho_0 \omega (-\sin(\omega t )\hat{x} + \cos(\omega t ) \hat{y}) + v_z \hat{z} \\
\dot{\vec{r}}_{(t)} &= \varrho_0 \omega \hat{\varphi} + v_z \hat{z} \\
\end{align*}

\subsection*{Aufgabe 2c}
$\dot{\vec{r}}_{(t)}$ nach der Zeit ableiten.
\begin{align*}
\ddot{\vec{r}}_{(t)} &= - \varrho_0 \omega^2 (\cos(\omega t )\hat{x} + \sin(\omega t ) \hat{y}) \\
\ddot{\vec{r}}_{(t)} &= - \varrho_0 \omega^2 \hat{\varrho} \\
\end{align*}


\section*{Aufgabe 4}
  \subsection*{Aufgbe 4a}
  Für die Übersicht wird $\pdv{x}{q_1}$ als $a$, $\pdv{x}{q_2}$ als $b$, etc. dargestellt.
  \begin{align*}
  \det \left(\begin{array}{c c c}
    a & b & c \\
    d & e & f \\
    g & h & i \\
  \end{array}\right) &= a \begin{vmatrix} e & f \\ h & i \end{vmatrix} - b \begin{vmatrix} d & f \\ g & i \end{vmatrix} + c \begin{vmatrix} d & e \\ g & h\end{vmatrix} \\
  &= aei - ahf - bdi + bgf + cdh - cge \\ \\
  \begin{pmatrix} a \\ d \\ g \end{pmatrix} \cdot \begin{pmatrix} \begin{pmatrix} b \\ e \\ h \end{pmatrix} \times \begin{pmatrix} c \\ f \\ i \end{pmatrix} \end{pmatrix} &= \begin{pmatrix} a \\ d \\ g \end{pmatrix} \cdot \begin{pmatrix} ei - hf \\ hc - bi \\ bf - ec \end{pmatrix} \\
  &= aei - ahf + dhc - dbi + gbf - gec
  \end{align*}
  Die beiden Ausdrücke sind gleich, was zu zeigen war.

  \subsection*{Aufgabe 4b}
  Volumenelement in Zylinderkoordinaten:
  \begin{align*}
  D &= \det \left(\begin{array}{c c c}
    \pdv{x}{\varrho} & \pdv{x}{\varphi} & \pdv{x}{z} \\
    \pdv{y}{\varrho} & \pdv{y}{\varphi} & \pdv{y}{z} \\
    \pdv{z}{\varrho} & \pdv{z}{\varphi} & \pdv{z}{z} \\
  \end{array}\right) \\
  &= \det \left(\begin{array}{c c c}
    \cos(\varphi) & -\varphi \sin(\varphi) & 0 \\
    \sin(\varphi) & \varrho \cos(\varphi) & 0 \\
    0 & 0 & 1 \\
  \end{array}\right) \\
  &= \varrho \cos^2(\varphi) + \varrho \sin^2(\varphi) = \varrho \\ \\
  dD &= \varrho \,d\varrho \, d\varphi \, dz
  \end{align*}

  Volumenelement in Kugelkoordinaten:
  \begin{align*}
  D &= \det \left(\begin{array}{c c c}
    \pdv{x}{\varrho} & \pdv{x}{\theta} & \pdv{x}{\varphi} \\
    \pdv{y}{\varrho} & \pdv{y}{\theta} & \pdv{y}{\varphi} \\
    \pdv{z}{\varrho} & \pdv{z}{\theta} & \pdv{z}{\varphi} \\
  \end{array}\right) \\
  &= \det \left(\begin{array}{c c c}
    \sin(\theta)\cos(\varphi) & \varrho \cos(\theta) \cos(\varphi) & - \varrho \sin(\theta) \sin(\varphi) \\
    \sin(\theta) \sin(\varphi) & \varrho \cos(\theta) \sin(\varphi) & \varrho \sin(\theta) \cos(\varphi) \\
    \cos(\theta) & - \varrho \sin(\theta) & 0 \\
  \end{array}\right) \\
  &= \sin(\theta) \cos(\varphi) (- \varrho^2 \sin^2(\theta) \cos(\varphi)) \\
  & \quad - \varrho \cos(\theta) \cos(\varphi) (\varrho \cos(\theta) \sin(\theta) \cos(\varphi)) \\ 
  & \quad - \varrho^2 \sin(\theta) \sin^2(\varphi) \\
  &= - \varrho^2 \left(\sin^3(\theta) \cos^2(\varphi) + \sin(\theta) \cos^2(\theta) \cos^2(\varphi) + \sin(\theta) \sin^2(\varphi)\right) \\ 
  &= - \varrho^2 \sin(\theta) \left( \sin^2(\theta) \cos^2(\varphi) + \cos^2(\theta) \cos^2(\varphi) + \sin^2(\varphi) \right) \\
  &= - \varrho^2 \sin(\theta) \left( \cos^2(\varphi) + \sin^2(\varphi) \right) \\
  &= - \varrho^2 \sin(\theta) \\
  &\mbox{Da man im Volumenelement den Betrag nimmt, folgt} \\ \\
  dD &= r^2 \sin(\theta) \,d\varrho \, d\varphi \, dz
  \end{align*}
\end{document}