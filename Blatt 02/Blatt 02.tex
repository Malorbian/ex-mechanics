\documentclass[a4paper,11pt]{article}
\usepackage[T1]{fontenc}
\usepackage[utf8]{inputenc}
\usepackage{lmodern}
\usepackage{ngerman}
\usepackage{amsmath}
\usepackage{graphicx}

\setcounter{secnumdepth}{0}
\title{Übungsblatt 01}
\author{Lennard Behrens, Tizian Roth}

\begin{document}

\maketitle

\section*{Aufgabe 1}
Die maximale Flughöhe während des Parabelflugs sei $x_1$ und die ursprüngliche Flughöhe sei $x_0$. Zum Zeitpunkt $t=0$ hat das Flugzeug den Scheitelpunkt der Parabelbahn erreicht. Allgemein gilt für die Flughöhe $x$
\begin{align*}
  x(t) = x_1 - \frac{1}{2} g t^2
\end{align*} 
Nach $\Delta t = 20 \mbox{s}$ ist das Flugzeug von Hochpunkt zurück auf die ursprüngliche Höhe gefallen. 
\begin{align*}
  x_0 &= x_1 - \frac{1}{2} g (\Delta t)^2 \\
  x_1 - x_0 &= \frac{1}{2} g (\Delta t)^2 \\
  &= 2,0 \mbox{km}
\end{align*}

\section*{Aufgabe 2}
\subsection*{Aufgabe 2a}
Seien $v_0$ die Abschussgeschwindigkeit und $\alpha$ der Abschusswinkel. So gilt für die Komponenten von $v_0$
\begin{align*}
  v_{0x} &= v_0 \cdot \cos\alpha\\
  v_{0y} &= v_0 \cdot \sin\alpha \mbox{.}
\end{align*}
Daraus folgt 
\begin{align*}
  x &= t \cdot v_0 \cos\alpha \\
  y &= t \cdot v_0 \sin\alpha - \frac{1}{2}g \cdot t^2 \\
  \Rightarrow t &= \frac{x}{v_0 \cos\alpha} \mbox{.}
\end{align*}
Nun kann man $t$ eliminieren.
\begin{align*}
  y(\alpha) &= x \cdot\tan\alpha - \frac{g \cdot x^2}{2v_0^2\cdot\cos^2\alpha} \\
  &= x \cdot\left(\tan\alpha - \frac{g \cdot x}{2v_0^2\cdot\cos^2\alpha}\right)
\end{align*}
Nun bestimmt man die Nullstellen dieser Funktion $y(\alpha)$, um die zurückgelegte Distanz zu erhalten.
\begin{align*}
  y(\alpha) &= x\cdot\left(\tan\alpha - \frac{g \cdot x}{2v_0^2\cdot\cos^2\alpha}\right) \\
  \Rightarrow x_1 &= 0 \\
  0 &=\tan\alpha - \frac{g \cdot x}{2v_0^2\cdot\cos^2\alpha} \\
  x_2(\alpha) &= 2\frac{v_0^2}{g} \cdot\tan\alpha\cos^2\alpha
\end{align*}
Da $x_1=0$ gibt $x_2(\alpha)$ die maximale Schussentfernung an. Um das Maximum für $x_2(\alpha)$ zu ermittelt, setzt man 
\begin{align*}
  \frac{dx_2}{d\alpha} &= 0\\
  0 &= 2\frac{v_0^2}{g}\cdot\left(1 - 2\cos\alpha\sin\alpha\tan\alpha\right)\\
  &= 2\frac{v_0^2}{g} \cdot\left(1 - 2\cdot\sin^2\alpha\right)\\
  1 &= 2\cdot\sin^2\alpha \\
  \alpha &= \arcsin\sqrt{\frac{1}{2}} = \frac{\pi}{4}=45^\circ
\end{align*}

\subsection*{Aufgabe 2b}
Aus Aufgabe 2a folgt für die Schussentfernung $d$
\begin{align*}
  d = 2\frac{v_0^2}{g} \cos^2\alpha\tan\alpha
\end{align*}
Da $\alpha = \frac{\pi}{4}$ gilt 
\begin{align*}
  d &= 2\frac{v_0^2}{g} \cos^2\alpha\\
  v_0 &= \sqrt{\frac{d\cdot g}{\cos^2\alpha}} = 28 \mbox{m/s}
\end{align*}

\subsection*{Aufgabe 2c}
Für die mittlere Geschwindigkeit gilt
\begin{align*}
  \overline{v} = \frac{\Delta x}{\Delta t} \Rightarrow t = \frac{\Delta x}{\overline{v}}
\end{align*}
Zu Beginn des Abbremsens hat der Ball die Geschwindigkeit $v = v_0$ und zum Ende $v = 0$. Daraus ergibt sich die mittlere Geschwindigkeit
\begin{align*}
  \overline{v} = \frac{1}{2} v_0 \mbox{.}
\end{align*}
Daraus folgt
\begin{align*}
  \Delta t = \frac{\Delta x}{\overline{v}} = 2\frac{\Delta x}{v_0} = 3,6 \mbox{ms.}
\end{align*}

\subsection*{Aufgabe 2d}
Für die mittlere Beschleunigung gilt
\begin{align*}
  \overline{a} = \frac{\Delta v}{\Delta t}
\end{align*}
Da die Geschwindigkeit $v_0$ gänzlich aufgehoben wird, gilt
\begin{align*}
  \overline{a} = \frac{v_0}{\Delta t} = \frac{v_0^2}{2\cdot\Delta x} = 7,8\cdot 10^3 \mbox{m/s}^2\mbox{.}
\end{align*}






\end{document}
